%% Generated by Sphinx.
\def\sphinxdocclass{report}
\documentclass[letterpaper,10pt,brazil]{sphinxmanual}
\ifdefined\pdfpxdimen
   \let\sphinxpxdimen\pdfpxdimen\else\newdimen\sphinxpxdimen
\fi \sphinxpxdimen=.75bp\relax

\PassOptionsToPackage{warn}{textcomp}
\usepackage[utf8]{inputenc}
\ifdefined\DeclareUnicodeCharacter
% support both utf8 and utf8x syntaxes
  \ifdefined\DeclareUnicodeCharacterAsOptional
    \def\sphinxDUC#1{\DeclareUnicodeCharacter{"#1}}
  \else
    \let\sphinxDUC\DeclareUnicodeCharacter
  \fi
  \sphinxDUC{00A0}{\nobreakspace}
  \sphinxDUC{2500}{\sphinxunichar{2500}}
  \sphinxDUC{2502}{\sphinxunichar{2502}}
  \sphinxDUC{2514}{\sphinxunichar{2514}}
  \sphinxDUC{251C}{\sphinxunichar{251C}}
  \sphinxDUC{2572}{\textbackslash}
\fi
\usepackage{cmap}
\usepackage[T1]{fontenc}
\usepackage{amsmath,amssymb,amstext}
\usepackage{babel}



\usepackage{times}
\expandafter\ifx\csname T@LGR\endcsname\relax
\else
% LGR was declared as font encoding
  \substitutefont{LGR}{\rmdefault}{cmr}
  \substitutefont{LGR}{\sfdefault}{cmss}
  \substitutefont{LGR}{\ttdefault}{cmtt}
\fi
\expandafter\ifx\csname T@X2\endcsname\relax
  \expandafter\ifx\csname T@T2A\endcsname\relax
  \else
  % T2A was declared as font encoding
    \substitutefont{T2A}{\rmdefault}{cmr}
    \substitutefont{T2A}{\sfdefault}{cmss}
    \substitutefont{T2A}{\ttdefault}{cmtt}
  \fi
\else
% X2 was declared as font encoding
  \substitutefont{X2}{\rmdefault}{cmr}
  \substitutefont{X2}{\sfdefault}{cmss}
  \substitutefont{X2}{\ttdefault}{cmtt}
\fi


\usepackage[Sonny]{fncychap}
\ChNameVar{\Large\normalfont\sffamily}
\ChTitleVar{\Large\normalfont\sffamily}
\usepackage{sphinx}

\fvset{fontsize=\small}
\usepackage{geometry}

% Include hyperref last.
\usepackage{hyperref}
% Fix anchor placement for figures with captions.
\usepackage{hypcap}% it must be loaded after hyperref.
% Set up styles of URL: it should be placed after hyperref.
\urlstyle{same}

\usepackage{sphinxmessages}
\setcounter{tocdepth}{1}



\title{UiPath}
\date{abr 27, 2020}
\release{}
\author{Paulo}
\newcommand{\sphinxlogo}{\vbox{}}
\renewcommand{\releasename}{}
\makeindex
\begin{document}

\ifdefined\shorthandoff
  \ifnum\catcode`\=\string=\active\shorthandoff{=}\fi
  \ifnum\catcode`\"=\active\shorthandoff{"}\fi
\fi

\pagestyle{empty}
\sphinxmaketitle
\pagestyle{plain}
\sphinxtableofcontents
\pagestyle{normal}
\phantomsection\label{\detokenize{index::doc}}



\chapter{Documentação UiPath}
\label{\detokenize{index:documentacao-uipath}}

\section{Guia para a Interface}
\label{\detokenize{index:guia-para-a-interface}}
Aqui você encontrará um guia de uso da interface do UiPath.


\section{Exercícios}
\label{\detokenize{index:exercicios}}
Aqui estão os exercícios que acompanham a documentação.


\section{Sumário}
\label{\detokenize{index:sumario}}

\subsection{Guia para a Interface}
\label{\detokenize{interface_guide:guia-para-a-interface}}\label{\detokenize{interface_guide::doc}}

\subsubsection{\sphinxstylestrong{I. O que é RPA?}}
\label{\detokenize{interface_guide:i-o-que-e-rpa}}\begin{quote}

RPA vem de Robotic Process Automation, que traduzido para o português torna-se “Automação Robótica de Processos”.
\end{quote}

\sphinxstylestrong{Automação}
\begin{quote}

Automação ocorre quando uma tarefa acontece automaticamente, ie. \sphinxstylestrong{sem intervenção humana}.
\end{quote}

\sphinxstylestrong{Robótica}
\begin{quote}

Um robô é uma entidade capaz de ser programada por um computador para realizar tarefas complexas. Em termos de ARP, esta tarefa seria \sphinxstylestrong{imitar ações humanas}.
\end{quote}
\begin{description}
\item[{\sphinxstylestrong{de Processos}}] \leavevmode
Um processo é uma \sphinxstylestrong{sequência de passos}, que levam a uma atividade ou tarefa significativa.

\end{description}


\subsubsection{\sphinxstylestrong{II. Começando no UiPath}}
\label{\detokenize{interface_guide:ii-comecando-no-uipath}}

\paragraph{Conceitos-chave}
\label{\detokenize{interface_guide:conceitos-chave}}\begin{description}
\item[{\sphinxstylestrong{Atividades}}] \leavevmode
Uma atividade é a menor ação possível no UIPath. Por exemplo, clicar com o botão esquerdo do mouse.

\item[{\sphinxstylestrong{Sequências}}] \leavevmode
Uma sequência é uma série de atividades que, em conjunto, realizam uma tarefa significativa. Por exemplo, entrar no seu e-mail.

\end{description}


\paragraph{\sphinxstylestrong{2.1 - Tipos de Projetos}}
\label{\detokenize{interface_guide:tipos-de-projetos}}\begin{description}
\item[{\sphinxstylestrong{Blank - Em branco}}] \leavevmode
Um projeto em branco é uma “tábula rasa” em que você pode construir seus projetos do zero.

\item[{\sphinxstylestrong{Simple Process - Processo Simples}}] \leavevmode
Um processo simples basicamente nos dá um modelo de um fluxograma ie. um diagrama de uma sequência de atividades.

\item[{\sphinxstylestrong{Agent Process Improvement - Melhoria de Processos de Agente}}] \leavevmode
Auxilia o usuário a automatizar as tarefas.

\end{description}

eg. configurar atalhos para cada tarefa
\begin{description}
\item[{\sphinxstylestrong{Transactional Business Process - Processo de Negócios Transacional}}] \leavevmode
É utilizado para definir estados em um projeto, que são úteis em processos de negócio.

\end{description}


\paragraph{\sphinxstylestrong{2.2 \textendash{} Componentes do UIPath}}
\label{\detokenize{interface_guide:componentes-do-uipath}}

\paragraph{\sphinxstylestrong{2.2.1 - Ribbon / Barra de Navegação}}
\label{\detokenize{interface_guide:ribbon-barra-de-navegacao}}

\paragraph{Recording \textendash{} Gravação}
\label{\detokenize{interface_guide:recording-gravacao}}\begin{quote}

O gravador do UIPath permite aos usuários gravar movimentos do mouse e atividades do teclado dentro da interface deu suário para gerar scripts de automação.
\end{quote}

\noindent\sphinxincludegraphics{{ribbon_recording}.png}

Como utilizar:
\begin{itemize}
\item {} 
Clique no botão

\item {} 
Realize a atividade que você quer automatizar

\item {} 
Pressione o botão ‘ESC’

\end{itemize}


\paragraph{Scraping - Aquisição de Dados}
\label{\detokenize{interface_guide:scraping-aquisicao-de-dados}}\begin{quote}

É utilizado para extrair informações da tela ou de dados provenientes de uma fonte externa.
\end{quote}

\noindent\sphinxincludegraphics{{ribbon_scraping}.png}


\paragraph{User Events - Eventos de Usuário}
\label{\detokenize{interface_guide:user-events-eventos-de-usuario}}\begin{quote}

Captura entradas que o usuário dá à interface e executa a ação. As entradas podem ser clicks do mouse, teclas pressionadas, rolagens de tela, etc.
\end{quote}

\noindent\sphinxincludegraphics{{ribbon_events}.png}


\paragraph{Remove Unused Variables - Remover Variáveis Não-utilizadas}
\label{\detokenize{interface_guide:remove-unused-variables-remover-variaveis-nao-utilizadas}}\begin{quote}

É utilizado para remover variáveis inutilizadas. Elas podem ser utilizadas caso exista um valor que mude durante o fluxo do programa ou para passar seu valor para outro componente.
\end{quote}

\noindent\sphinxincludegraphics{{ribbon_remove_variables}.png}


\paragraph{New \textendash{} Novo}
\label{\detokenize{interface_guide:new-novo}}
\begin{figure}[htbp]
\centering
\capstart

\noindent\sphinxincludegraphics{{ribbon_new}.png}
\caption{Utilizado para criar arquivos dentro do projeto.}\label{\detokenize{interface_guide:id1}}\end{figure}


\paragraph{Sequência}
\label{\detokenize{interface_guide:sequencia}}
\begin{figure}[htbp]
\centering
\capstart

\noindent\sphinxincludegraphics{{ribbon_new_sequence}.png}
\caption{Cria uma nova sequência em branco.}\label{\detokenize{interface_guide:id2}}\end{figure}


\paragraph{Fluxograma}
\label{\detokenize{interface_guide:fluxograma}}
\begin{figure}[htbp]
\centering

\noindent\sphinxincludegraphics{{ribbon_new_flowchart}.png}
\end{figure}

Cria um novo modelo de fluxograma.


\paragraph{State Machine - Máquina de Evento}
\label{\detokenize{interface_guide:state-machine-maquina-de-evento}}
\begin{figure}[htbp]
\centering
\capstart

\noindent\sphinxincludegraphics{{ribbon_new_state_machine}.png}
\caption{Cria um novo modelo de máquina de estado.}\label{\detokenize{interface_guide:id3}}\end{figure}


\paragraph{\sphinxstylestrong{2.2.2 \textendash{} Painel de Atividades}}
\label{\detokenize{interface_guide:painel-de-atividades}}\begin{quote}

O painel de atividades exibe todas as atividades disponíveis para uso na automação de tarefas.
\end{quote}


\paragraph{\sphinxstylestrong{Computer Vision - Visão Computacional}}
\label{\detokenize{interface_guide:computer-vision-visao-computacional}}
\begin{figure}[htbp]
\centering
\capstart

\noindent\sphinxincludegraphics{{activity_cv}.png}
\caption{Contém atividades relacionadas à algoritmos de visão computacional.}\label{\detokenize{interface_guide:id4}}\end{figure}


\paragraph{\sphinxstylestrong{Orchestrator - Orquestrador}}
\label{\detokenize{interface_guide:orchestrator-orquestrador}}
\begin{figure}[htbp]
\centering
\capstart

\noindent\sphinxincludegraphics{{activity_orchestrator}.png}
\caption{Gerenciamento de atividades distribuídas entre vários computadores.}\label{\detokenize{interface_guide:id5}}\end{figure}


\paragraph{\sphinxstylestrong{Programming - Programação}}
\label{\detokenize{interface_guide:programming-programacao}}
\begin{figure}[htbp]
\centering
\capstart

\noindent\sphinxincludegraphics{{activity_programming}.png}
\caption{Aqui estão atividades destinadas a problemas comuns em programação, como fluxo de controle, tratamento de exceções, etc.}\label{\detokenize{interface_guide:id6}}\end{figure}


\paragraph{\sphinxstylestrong{System - Sistema}}
\label{\detokenize{interface_guide:system-sistema}}
\begin{figure}[htbp]
\centering
\capstart

\noindent\sphinxincludegraphics{{activity_cv}.png}
\caption{Atividades relacionadas ao sistema, eg. copiar para a área de transferência, deletar arquivo.}\label{\detokenize{interface_guide:id7}}\end{figure}


\paragraph{\sphinxstylestrong{Testing - Testes}}
\label{\detokenize{interface_guide:testing-testes}}
\begin{figure}[htbp]
\centering
\capstart

\noindent\sphinxincludegraphics{{activity_testing}.png}
\caption{Atividades relacionadas a testes internos da atividade.}\label{\detokenize{interface_guide:id8}}\end{figure}


\paragraph{\sphinxstylestrong{UI Automation - Automação de Interface de Usuário}}
\label{\detokenize{interface_guide:ui-automation-automacao-de-interface-de-usuario}}
\begin{figure}[htbp]
\centering
\capstart

\noindent\sphinxincludegraphics{{activity_uiautomation}.png}
\caption{Nesta seção estão contidas atividades concernentes à automação de eventos da interface.}\label{\detokenize{interface_guide:id9}}\end{figure}
\begin{itemize}
\item {} 
\sphinxstylestrong{Element - Elemento}

\end{itemize}
\begin{quote}

Eventos de mouse, eventos de teclado, eventos de texto.
\end{quote}
\begin{itemize}
\item {} 
\sphinxstylestrong{Image - Imagem}

\end{itemize}
\begin{quote}

Eventos de imagem.
\end{quote}
\begin{itemize}
\item {} 
\sphinxstylestrong{OCR \textendash{} Optical Character Recognition \textendash{} Reconhecimento Óptico de Caracteres}

\end{itemize}
\begin{quote}

Extrair texto de imagens.
\end{quote}
\begin{itemize}
\item {} 
\sphinxstylestrong{User Events - Eventos de Usuário}

\end{itemize}
\begin{quote}

Gatilhos do usuário.
\end{quote}
\begin{itemize}
\item {} 
\sphinxstylestrong{Workflow - Fluxo de Trabalho}

\end{itemize}
\begin{quote}

Atividades que estão dentro de fluxogramas. Geralmente envolvem tomada de decisão por parte do sistema.
\end{quote}


\paragraph{\sphinxstylestrong{2.3 - Painel de Propriedades}}
\label{\detokenize{interface_guide:painel-de-propriedades}}
\begin{figure}[htbp]
\centering

\noindent\sphinxincludegraphics{{property_panel}.png}
\end{figure}

Mostra as principais propriedades para a atividade atualmente selecionada.
\begin{description}
\item[{\sphinxstylestrong{Variáveis} - variáveis definidas para a atividade atual.}] \leavevmode
\begin{figure}[htbp]
\centering

\noindent\sphinxincludegraphics{{property_variables}.png}
\end{figure}

\item[{\sphinxstylestrong{Argumentos} - podem ser passados para outros arquivos enquanto variáveis não podem.}] \leavevmode
\begin{figure}[htbp]
\centering

\noindent\sphinxincludegraphics{{property_arguments}.png}
\end{figure}

\item[{\sphinxstylestrong{Importações} - lista todas as atividades que serão importadas por padrão quando você tentar criar seu programa.}] \leavevmode
\begin{figure}[htbp]
\centering

\noindent\sphinxincludegraphics{{property_imports}.png}
\end{figure}

\end{description}


\subsection{Exercícios}
\label{\detokenize{exercises:exercicios}}\label{\detokenize{exercises::doc}}

\subsubsection{Exercício I}
\label{\detokenize{exercises:exercicio-i}}
Neste exemplo criaremos uma sequência que captura o nome do usuário a partir de um Diálogo de Entrada  - \sphinxstylestrong{Input Dialog} e exibe uma mensagem \sphinxstylestrong{“Olá, {[}nome da entrada{]}”} com o nome da entrada dada em uma Caixa de Mensagem - \sphinxstylestrong{Message Box}.

{\hyperref[\detokenize{exercise_1::doc}]{\sphinxcrossref{\DUrole{doc}{Exercicio I}}}}


\subsubsection{Exercício II}
\label{\detokenize{exercises:exercicio-ii}}
Neste exercício, iremos realizar scraping de uma página web \textendash{} fakenamegenerator.com \textendash{} 50 vezes, salvar os dados em uma planilha do Excel e mandar a planilha para um e-mail especificado.

{\hyperref[\detokenize{exercise_2::doc}]{\sphinxcrossref{\DUrole{doc}{Exercicio II}}}}


\subsection{Exercício I}
\label{\detokenize{exercise_1:exercicio-i}}\label{\detokenize{exercise_1::doc}}

\subsubsection{\sphinxstylestrong{Exercício I}}
\label{\detokenize{exercise_1:id1}}

\paragraph{Criar uma tela que exibe o nome digitado}
\label{\detokenize{exercise_1:criar-uma-tela-que-exibe-o-nome-digitado}}
Neste exemplo criaremos uma sequência que captura o nome do usuário a partir de um Diálogo de Entrada  - \sphinxstylestrong{Input Dialog} e exibe uma mensagem \sphinxstylestrong{“Olá, {[}nome da entrada{]}”} com o nome da entrada dada em uma Caixa de Mensagem - \sphinxstylestrong{Message Box}.


\paragraph{\sphinxstylestrong{I \textendash{} Crie uma nova sequência}}
\label{\detokenize{exercise_1:i-crie-uma-nova-sequencia}}
Crie uma nova sequência clicando em \sphinxstylestrong{New -\textgreater{} Sequence}.

\begin{figure}[htbp]
\centering

\noindent\sphinxincludegraphics{{ex1_new_sequence}.png}
\end{figure}


\paragraph{\sphinxstylestrong{II - Escolha o nome da nova sequência}}
\label{\detokenize{exercise_1:ii-escolha-o-nome-da-nova-sequencia}}
Escolha o nome da nova sequência no diálogo que aparecerá.

\begin{figure}[htbp]
\centering

\noindent\sphinxincludegraphics{{ex1_name_sequence}.png}
\end{figure}


\paragraph{\sphinxstylestrong{III \textendash{} Sequência vazia}}
\label{\detokenize{exercise_1:iii-sequencia-vazia}}
Aparecerá uma caixa contendo a sequência vazia.

\begin{figure}[htbp]
\centering

\noindent\sphinxincludegraphics{{ex1_drop_sequence}.png}
\end{figure}


\paragraph{\sphinxstylestrong{IV \textendash{} Insira Input Dialog na sequência}}
\label{\detokenize{exercise_1:iv-insira-input-dialog-na-sequencia}}
Arraste o Input Dialog para a sequência, contido em \sphinxstylestrong{System -\textgreater{} Dialog -\textgreater{} Input Dialog}.

\begin{figure}[htbp]
\centering

\noindent\sphinxincludegraphics{{ex1_add_input_dialog}.png}
\end{figure}


\paragraph{\sphinxstylestrong{V \textendash{} Preencha o Input Dialog}}
\label{\detokenize{exercise_1:v-preencha-o-input-dialog}}
Preencha o título da janela (“Title”) e o texto a ser exibido acima da caixa de entrada (“Label”)

\sphinxstylestrong{OBS}: Todo texto que não seja nome de variável ou expressão deve estar contido entre aspas.

\begin{figure}[htbp]
\centering

\noindent\sphinxincludegraphics{{ex1_fill_input_dialog}.png}
\end{figure}


\paragraph{\sphinxstylestrong{VI \textendash{} Crie uma nova variável para guardar o valor da entrada}}
\label{\detokenize{exercise_1:vi-crie-uma-nova-variavel-para-guardar-o-valor-da-entrada}}
Crie uma nova variável para guardar o valor da entrada do \sphinxstylestrong{Input Dialog} - no nosso caso a variável chama-se “name”.

\begin{figure}[htbp]
\centering

\noindent\sphinxincludegraphics{{ex1_create_variable}.png}
\end{figure}


\paragraph{\sphinxstylestrong{VII \textendash{} Configure a saída do Input Dialog}}
\label{\detokenize{exercise_1:vii-configure-a-saida-do-input-dialog}}
Ainda com o Input Dialog selecionado, clique na aba Properties e defina o valor “Result” com o nome da variável que você selecionou - no nosso caso “name”.

\begin{figure}[htbp]
\centering

\noindent\sphinxincludegraphics{{ex1_configure_output}.png}
\end{figure}


\paragraph{\sphinxstylestrong{VII \textendash{} Insira Message Box na sequência}}
\label{\detokenize{exercise_1:vii-insira-message-box-na-sequencia}}
Arraste a \sphinxstylestrong{Message Box} para a sequência, contida em System -\textgreater{} Dialog -\textgreater{} Message Box.

\begin{figure}[htbp]
\centering

\noindent\sphinxincludegraphics{{ex1_add_message_box}.png}
\end{figure}


\paragraph{\sphinxstylestrong{VIII \textendash{} Inserir Texto na Message Box}}
\label{\detokenize{exercise_1:viii-inserir-texto-na-message-box}}
Insira o texto a ser exibido (no nosso caso “Ola, “) concatenado com o nome da variável (+ name).

\begin{figure}[htbp]
\centering

\noindent\sphinxincludegraphics{{ex1_fill_message_box}.png}
\end{figure}


\paragraph{\sphinxstylestrong{IX \textendash{} Execute a sequência}}
\label{\detokenize{exercise_1:ix-execute-a-sequencia}}
Aperte F6 para rodar o programa ou clique no botão Debug File dentro do Ribbon.

\begin{figure}[htbp]
\centering

\noindent\sphinxincludegraphics{{ex1_ribbon_debug}.png}
\end{figure}


\paragraph{\sphinxstylestrong{X \textendash{} Programa sendo Executado}}
\label{\detokenize{exercise_1:x-programa-sendo-executado}}
O programa irá rodar e aparecerá uma caixa de diálogo pedindo para você inserir seu nome.
\begin{quote}

\begin{figure}[htbp]
\centering

\noindent\sphinxincludegraphics{{ex1_insert_input}.png}
\end{figure}
\end{quote}


\paragraph{\sphinxstylestrong{XI \textendash{} Diálogo Final}}
\label{\detokenize{exercise_1:xi-dialogo-final}}
Logo em seguida aparecerá uma caixa contendo o diálogo com o nome que você inseriu anteriormente.
\begin{quote}

\begin{figure}[htbp]
\centering

\noindent\sphinxincludegraphics{{ex1_final_dialog}.png}
\end{figure}
\end{quote}


\subsection{Exercício II}
\label{\detokenize{exercise_2:exercicio-ii}}\label{\detokenize{exercise_2::doc}}

\subsubsection{\sphinxstylestrong{Exercício II}}
\label{\detokenize{exercise_2:id1}}

\paragraph{Capturar dados de uma página web e salvá-los em uma planilha do Excel}
\label{\detokenize{exercise_2:capturar-dados-de-uma-pagina-web-e-salva-los-em-uma-planilha-do-excel}}
Neste exercício, iremos realizar scraping de uma página web \textendash{} fakenamegenerator.com \textendash{} 50 vezes e salva os dados em uma planilha do Excel.


\paragraph{\sphinxstylestrong{I \textendash{} Crie uma nova sequência}}
\label{\detokenize{exercise_2:i-crie-uma-nova-sequencia}}
Crie uma nova sequência clicando em \sphinxstylestrong{New -\textgreater{} Sequence}.

\begin{figure}[htbp]
\centering

\noindent\sphinxincludegraphics{{ex1_new_sequence}.png}
\end{figure}


\paragraph{\sphinxstylestrong{II - Escolha o nome da nova sequência}}
\label{\detokenize{exercise_2:ii-escolha-o-nome-da-nova-sequencia}}
Escolha o nome da nova sequência no diálogo que aparecerá.

\begin{figure}[htbp]
\centering

\noindent\sphinxincludegraphics{{ex1_name_sequence}.png}
\end{figure}


\paragraph{\sphinxstylestrong{III \textendash{} Sequência vazia}}
\label{\detokenize{exercise_2:iii-sequencia-vazia}}
Aparecerá uma caixa contendo a sequência vazia.

\begin{figure}[htbp]
\centering

\noindent\sphinxincludegraphics{{ex1_drop_sequence}.png}
\end{figure}


\paragraph{\sphinxstylestrong{IV \textendash{} Insira Open Browser na sequência}}
\label{\detokenize{exercise_2:iv-insira-open-browser-na-sequencia}}
Arraste Open Browser para a sequência, contido em \sphinxstylestrong{UI Automation -\textgreater{} Open Browser}.

\begin{figure}[htbp]
\centering

\noindent\sphinxincludegraphics{{ex2_add_open_browser}.png}
\end{figure}


\paragraph{\sphinxstylestrong{V \textendash{} Selecionar navegador}}
\label{\detokenize{exercise_2:v-selecionar-navegador}}
Clique na aba Properties e selecione o navegador a ser utilizado.

\begin{figure}[htbp]
\centering

\noindent\sphinxincludegraphics{{ex2_choose_browser}.png}
\end{figure}


\paragraph{\sphinxstylestrong{VI \textendash{} Inserir URL}}
\label{\detokenize{exercise_2:vi-inserir-url}}
Insira a URL desejada (“\sphinxurl{http://www.fakenamegenerator.com}”, no nosso caso).

\begin{figure}[htbp]
\centering

\noindent\sphinxincludegraphics{{ex2_insert_url}.png}
\end{figure}


\paragraph{\sphinxstylestrong{VII \textendash{} Inserir Input Dialog na sequência}}
\label{\detokenize{exercise_2:vii-inserir-input-dialog-na-sequencia}}
Arraste um Input Dialog para a sequência, contido em System -\textgreater{} Dialog -\textgreater{} Input Dialog e preencha-o com o título “Enter details” e label “Entre com o número de consumidores”.

\begin{figure}[htbp]
\centering

\noindent\sphinxincludegraphics{{ex2_insert_dialog}.png}
\end{figure}


\paragraph{\sphinxstylestrong{VIII \textendash{} Criar variável que alocará número de ciclos}}
\label{\detokenize{exercise_2:viii-criar-variavel-que-alocara-numero-de-ciclos}}
Crie uma variável “numero” para especificar a quantidade de vezes que iremos procurar por dados na página - no nosso caso “number”.

\begin{figure}[htbp]
\centering

\noindent\sphinxincludegraphics{{ex2_add_cycles_variable}.png}
\end{figure}


\paragraph{\sphinxstylestrong{IX \textendash{} Configure a saída do Input Dialog}}
\label{\detokenize{exercise_2:ix-configure-a-saida-do-input-dialog}}
Abra a aba propriedades e coloque a variável “numero” como saída do Input Dialog.

\begin{figure}[htbp]
\centering

\noindent\sphinxincludegraphics{{ex2_configure_input_dialog}.png}
\end{figure}


\paragraph{\sphinxstylestrong{X \textendash{} Criar Loop Do While}}
\label{\detokenize{exercise_2:x-criar-loop-do-while}}
Arraste a atividade Do While para a sequência dentro da atividade Do, do Open Browser. Está contida em \sphinxstylestrong{Workflow -\textgreater{} Control -\textgreater{} Do While}.

\begin{figure}[htbp]
\centering

\noindent\sphinxincludegraphics{{ex2_add_do_while_loop}.png}
\end{figure}


\paragraph{\sphinxstylestrong{XI \textendash{} Adicionar Condição de Parada}}
\label{\detokenize{exercise_2:xi-adicionar-condicao-de-parada}}
Coloque a expressão “val \textless{} Cint(number)” como condição de parada (Condition) do ciclo.
\begin{quote}

\begin{figure}[htbp]
\centering

\noindent\sphinxincludegraphics{{ex2_add_stop_condition}.png}
\end{figure}
\end{quote}


\paragraph{\sphinxstylestrong{XII \textendash{} Inserir atividade Assign}}
\label{\detokenize{exercise_2:xii-inserir-atividade-assign}}
Arraste a atividade Assign, contida em \sphinxstylestrong{System -\textgreater{} Activities -\textgreater{} Statements}, para a sequência dentro do Body e assinale seu valor como \sphinxstylestrong{val = val + 1}.
\begin{quote}

\begin{figure}[htbp]
\centering

\noindent\sphinxincludegraphics{{ex2_add_assign}.png}
\end{figure}
\end{quote}


\paragraph{\sphinxstylestrong{XIII \textendash{} Inserir GetFullText}}
\label{\detokenize{exercise_2:xiii-inserir-getfulltext}}
Arraste a atividade Get Full Text, contida em \sphinxstylestrong{UI Automation -\textgreater{} Text -\textgreater{} Screen Scraping} para a sequência.

\begin{figure}[htbp]
\centering

\noindent\sphinxincludegraphics{{ex2_add_get_full_text}.png}
\end{figure}


\paragraph{\sphinxstylestrong{XIV \textendash{} Inserir Nome}}
\label{\detokenize{exercise_2:xiv-inserir-nome}}
Clique em ‘Indicate element inside browser’ e selecione o nome dentro da página do site Fake Name Generator

\begin{figure}[htbp]
\centering

\noindent\sphinxincludegraphics{{ex2_insert_name}.png}
\end{figure}


\paragraph{\sphinxstylestrong{XV \textendash{} Criar Variável Nome}}
\label{\detokenize{exercise_2:xv-criar-variavel-nome}}
Crie uma variável “nome” e especifique-a como saída da atividade Get Full Text.

\begin{figure}[htbp]
\centering

\noindent\sphinxincludegraphics{{ex2_add_variable_name}.png}
\end{figure}


\paragraph{\sphinxstylestrong{XVI \textendash{} Repertir Passos XIV e XV}}
\label{\detokenize{exercise_2:xvi-repertir-passos-xiv-e-xv}}
Repita os passos XIV e XV mas para as variáveis “telefone” e “datanascimento”.


\paragraph{\sphinxstylestrong{XVII \textendash{} Criar DataTable}}
\label{\detokenize{exercise_2:xvii-criar-datatable}}
Agora criaremos uma tabela de dados, para isso arraste a atividade Build Data Table, contida em \sphinxstylestrong{Programming -\textgreater{} DataTable}, à sequência, logo abaixo de Input Dialog.

\begin{figure}[htbp]
\centering

\noindent\sphinxincludegraphics{{ex2_add_data_table}.png}
\end{figure}


\paragraph{\sphinxstylestrong{XVIII \textendash{} Preencher campos DataTable}}
\label{\detokenize{exercise_2:xviii-preencher-campos-datatable}}
Adicione os campos na tabela, clicando em DataTable… e preenchendo os campos

\begin{figure}[htbp]
\centering

\noindent\sphinxincludegraphics{{ex2_insert_data_table_fields}.png}
\end{figure}


\paragraph{\sphinxstylestrong{XIX \textendash{} Criar variável que irá conter DataTable}}
\label{\detokenize{exercise_2:xix-criar-variavel-que-ira-conter-datatable}}
Crie uma variável para guardar os resultados com o nome de “ExtracaoDataTable”.

\begin{figure}[htbp]
\centering

\noindent\sphinxincludegraphics{{ex2_add_data_table_variable}.png}
\end{figure}


\paragraph{\sphinxstylestrong{XX \textendash{} Alterar tipo da variável}}
\label{\detokenize{exercise_2:xx-alterar-tipo-da-variavel}}
Crie uma variável para guardar os resultados com o nome de “ExtracaoDataTable”.

\begin{figure}[htbp]
\centering

\noindent\sphinxincludegraphics{{ex2_change_var_type}.png}
\end{figure}

\begin{figure}[htbp]
\centering

\noindent\sphinxincludegraphics{{ex2_change_var_type_2}.png}
\end{figure}


\paragraph{\sphinxstylestrong{XXI \textendash{} Adicionar atividade AddDataRow}}
\label{\detokenize{exercise_2:xxi-adicionar-atividade-adddatarow}}
Arraste a atividade Add Data Row, contida em \sphinxstylestrong{Programming -\textgreater{} DataTable}, à sequência, logo abaixo de Get Full Text.

\begin{figure}[htbp]
\centering

\noindent\sphinxincludegraphics{{ex2_add_data_row}.png}
\end{figure}


\paragraph{\sphinxstylestrong{XXII \textendash{} Assinalar AddDataRow}}
\label{\detokenize{exercise_2:xxii-assinalar-adddatarow}}
Assinale à ArrayRow da atividade Add Data Row o dicionário contendo o nome, telefone e datanascimento (\{nome,telefone,datanascimento\}) e DataTable o nome da DataTable que definimos no passo XVIII.

\begin{figure}[htbp]
\centering

\noindent\sphinxincludegraphics{{ex2_assign_data_row}.png}
\end{figure}


\paragraph{\sphinxstylestrong{XXIII - Configurar Página para Atualizar}}
\label{\detokenize{exercise_2:xxiii-configurar-pagina-para-atualizar}}
Precisamos atualizar a página toda vez que quisermos um registro novo, faremos isso clicando no botão Generate, que gera outro registro. Para fazer isso, arraste a atividade Click, contida em \sphinxstylestrong{UI Automation -\textgreater{} Element -\textgreater{} Mouse} e defina o elemento da página desejado clicando em “Indicate element inside browser”. O elemento é o botão Generate na página.

\begin{figure}[htbp]
\centering

\noindent\sphinxincludegraphics{{ex2_click_generator}.png}
\end{figure}


\subsection{License}
\label{\detokenize{license:license}}\label{\detokenize{license::doc}}
MIT LICENSE

Permission is hereby granted, free of charge, to any person obtaining a copy of this software and associated documentation files (the “Software”), to deal in the Software without restriction, including without limitation the rights to use, copy, modify, merge, publish, distribute, sublicense, and/or sell copies of the Software, and to permit persons to whom the Software is furnished to do so, subject to the following conditions:

The above copyright notice and this permission notice shall be included in all copies or substantial portions of the Software.

THE SOFTWARE IS PROVIDED “AS IS”, WITHOUT WARRANTY OF ANY KIND, EXPRESS OR IMPLIED, INCLUDING BUT NOT LIMITED TO THE WARRANTIES OF MERCHANTABILITY, FITNESS FOR A PARTICULAR PURPOSE AND NONINFRINGEMENT. IN NO EVENT SHALL THE AUTHORS OR COPYRIGHT HOLDERS BE LIABLE FOR ANY CLAIM, DAMAGES OR OTHER LIABILITY, WHETHER IN AN ACTION OF CONTRACT, TORT OR OTHERWISE, ARISING FROM, OUT OF OR IN CONNECTION WITH THE SOFTWARE OR THE USE OR OTHER DEALINGS IN THE SOFTWARE.


\chapter{Índice e Tabelas}
\label{\detokenize{index:indice-e-tabelas}}\begin{itemize}
\item {} 
\DUrole{xref,std,std-ref}{genindex}

\item {} 
\DUrole{xref,std,std-ref}{modindex}

\item {} 
\DUrole{xref,std,std-ref}{search}

\end{itemize}



\renewcommand{\indexname}{Índice}
\printindex
\end{document}